% lengths

\newcommand{\short}{.2cm}
\newcommand{\veryshort}{.2mm}

\newcommand{\cornercut}{\short} % how much is missing at the corner of decorated inverted textfields

\newlength{\cslineheight}
\setlength{\cslineheight}{\short}
\addtolength{\cslineheight}{-\veryshort}
\newcommand{\shadowshift}{\veryshort}
\newcommand{\nodedistance}{\short}

\newlength{\lineheight}
\setlength{\lineheight}{1cm}
\newcommand{\attributewidth}{1cm}

\newlength{\attributelabelwidth}
\addtolength{\attributelabelwidth}{\attributewidth}
\addtolength{\attributelabelwidth}{\attributewidth}
\addtolength{\attributelabelwidth}{\attributewidth}
\addtolength{\attributelabelwidth}{\nodedistance}
\addtolength{\attributelabelwidth}{\nodedistance}

\newlength{\flufffield}
\addtolength{\flufffield}{\attributewidth}
\addtolength{\flufffield}{\attributewidth}
\addtolength{\flufffield}{\nodedistance}

\newlength{\namefield}
\addtolength{\namefield}{\flufffield}
\addtolength{\namefield}{\flufffield}
\addtolength{\namefield}{\flufffield}
\addtolength{\namefield}{-\nodedistance}

% yet unused
\newcommand{\picshift}{-1mm}
\newcommand{\textfieldsize}{1cm}
\newlength{\currentFieldWith}
\newlength{\currentFieldHeight}
\setlength{\baselineskip}{10pt}

\tikzset{
  node distance=\nodedistance,
}
